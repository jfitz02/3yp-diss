\chapter{Design}
\label{ch:design}

In this chapter, we describe the overall design of our solution to the problem identified in \Cref{ch:introduction}, building on work described in \Cref{ch:background}.

\section{Topic Classification}
Chapter \ref{ch:background} discussed 4 different models for classifying text into topics: RNN, LSTM, BERT, and RoBERTa.
The chapter also used intuition to determine which model would be expected to perform best - RoBERTa. The reason being,
RNNs and LSTMs suffer from being unable to understand context of a whole sentence; they are limited to understanding context
in a single direction (left to right or right to left). BERT and RoBERTa both use a technique called self-attention to overcome
this limitation. From observing results in \cite{DBLP:journals/corr/abs-1907-11692} RoBERTa outperforms BERT in most cases.\\
Building off of the RoBERTa model described in \Cref{ch:background}, we use the model to classify posts into topics.
The design of this model is relatively simple due to the fact we are performing transfer learning on a pre-trained model.
\begin{figure}[hbtp]
    \centering
    \includegraphics[width=0.6\textwidth]{../images/classification-model.png}
    \caption{RoBERTa model}
    \label{fig:roberta}
\end{figure}

As seen in the diagram, the RoBERTa model has been altered with a new classification layer.

\section{Adding Context}
\subsection{What is context? and why is it important?}
As discussed in section \rec{sec:pythia}, context is an important factor in classifying posts. In figure \ref{fig:tweet-example} there
is a tweet that would be hard to classify with the text alone. The section then goes onto show how adding the image that the tweet
was posted with (figure \ref{fig:tweet-image}) adds more information to the post and makes it easier to classify.
\subsection{Methods for adding context}
There are many ways to add context to a post. Pythia uses Named Entity Recognition (NER) for adding context. In this project the use
of Optical Character Recognition, Audio Transcription, and Threads/Retweets are explored.
\subsubsection{Named Entity Recognition - NER}
\subsubsection{Optical Character Recognition - OCR}
Images in posts may contain text. This text adds context to the post and can be helpful in classifying the post.
Using OCR we can extract any text from an image and use it in addition to the text of the post. This should improve the accuracy of the
model when infographics/text based images are used.
\subsubsection{Audio Transcription - Wav2Vec}
Some posts may contain videos. The audio in the videos may contain useful context. Using Wav2Vec we can extract the audio from the video
in the form of text. This text can be used in addition to the text of the post. This should improve the accuracy of the model when audio
descriptions/explanations are used in a post.
\subsubsection{Retweets and Threads}
Although a post alone may not contain enough context to classify it, there may be a conversation around the post. This conversation 
would be found in the retweets and threads of the post. If the retweets and threads are discussing the same topic as the post, then the
extra context given by the retweets and threads can be used to help classify the post.

\subsection{Context Aware Model}
The context aware model created in this project will use OCR, Wav2Vec, and retweets/threads to add context to posts. The extra text
extracted from these methods will be added to the text of the post. This will be done before the post is classified. The model will then
classified with all the additional text.
\begin{figure}
    \centering
    \includegraphics[width=0.6\textwidth]{../images/context-aware-model.png}
    \caption{Context Aware Model}
    \label{fig:context-aware-model}
\end{figure}

The RoBERTa classification model will be the same as the one designed figure \ref{fig:roberta}. The only difference is the input
to the model.
\section{Python Application}
The Python application is made to show users the topics they are interested in and allows them to compare their interests to the 
social media site.\\
The first step of design for the application was to decide what features the application should have. This led to the process of
requirements analysis.\\
\subsection{Requirements Analysis}
\subsubsection{User Stories}
Before building the application, user stories were created to help guide what features are required. The user stories created were:
\begin{enumerate}
    \item As a user, I want to be able to see what topics I see the most on social media/see what topics I am interested in.
    \item As a user, I want to be able to compare what I see on social media to what other people see on social media.
    \item As a user, I want to be able to reach out and find posts on topics I am not interested in.
    \item As a user, I want to be able to see what topics are trending on social media.
\end{enumerate}
These stories help set up the requirements for the application.
\subsubsection{Business Cases}
The next step is to outline the business cases (methods of solving the problem) for the application. For this project,
All business cases will be made by the author of this dissertation. The business cases created were:
\begin{enumerate}
    \item \textbf{Do Nothing} - This does not improve the problems users face. It is a baseline case.
    \item \textbf{Chrome Extension} - Allow users to see what percentage of their social media feed is made up of each topic, as well as compare this to live data from the social media platform.
    \item \textbf{Python Application} - Allow users to see what percentage of their social media feed is made up of each topic, as well as compare this to live data from the social media platform.
\end{enumerate}

The difference between the chrome extension and the python application is the framework they are built in as well as how
they are interacted with. The chrome extension would be built in JavaScript, whereas the python application would be built in Python.
The chrome extension would be accessible from a chrome browser (via the extensions store), whereas the python application would be
run from a python script.\\
Although, a chrome extension would be more accessible to users and easier to distribute, it would be more difficult to implement
as I would have to learn JavaScript and the chrome extension API. I am already familiar with Python and the python libraries used
in this project.
\subsubsection{Requirements}
Using the user stories and chosen business case, the requirements for the application can be determined. \textbf{C} - User Requirement
\textbf{D} - System Requirements
\begin{enumerate}
    \item Functional Requirements
    \begin{enumerate}
        \item C) The user should be able to see what topics they see the most on social media/see what topics they are interested in.
        \begin{enumerate}
            \item D) The application should be able to get access to the users social media feed via the social media API.
            \item D) The application should be able to classify the posts in the users social media feed.
            \item D) The application should be able to calculate the percentage of posts in the users social media feed that are about each topic.
            \item D) The application should display the top 5 topics the user is interested in as well as their percentage impact on the users social media feed.
        \end{enumerate}
        \item C) The user should be able to compare what they see on social media to what other people see on social media.
        \begin{enumerate}
            \item D) The application should be able to get access to live posts from the social media API.
            \item D) The application should be able to classify the live posts from the social media API.
            \item D) The application should be able to calculate the percentage of live posts that are about each topic.
            \item D) The application should display the top 5 topics that are trending on social media as well as their percentage impact on the social media platform.
            \item D) The application should display a similarity metric between the users social media feed and the live posts.
        \end{enumerate}
        \item C) The user should be able to reach out and find posts on topics they are not interested in.
        \begin{enumerate}
            \item D) The application should store all posts that are classified as a topic to be able to search through them.
            \item D) The application should store alongside the post the top topic it was classified as.
            \item D) The application should allow the user to search for posts by topic.
            \item D) The application should display a random selection of posts that are classified as the topic the user searched for.
        \end{enumerate}
    \end{enumerate}
    \item Non-Functional Requirements
    \begin{enumerate}
        \item C) The User Interface should be easy to use within 5 minutes of use.
        \item C) The User Interface should not be unresponsive for more than 5 seconds.
        \item C) The User Interface should be suitable for users with no technical experience.
    \end{enumerate}
\end{enumerate}

\subsection{Frontend}
\subsubsection{User Interface Design}
\subsection{Backend}
\subsubsection{API Design}
\subsubsection{Database Design}