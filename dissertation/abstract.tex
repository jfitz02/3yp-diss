\begin{abstract}
    Thanks to recommender systems it is common that a users social media feed will be filled with posts relating to other posts they have
    previously interacted with. This can lead to a user being shown posts that are similar to the posts they have already seen. In this project
    it is viewed that this can lead to a user being shown posts that are biased towards their own opinions. This project aims to identify topics
    that users are interested in based off of their social media feed.\\
    First, a comparison of different Natural Language Processing models is completed to identify what model is best suited for the task
    of classifying social media posts into topics (sports, politics, etc.). It was found that RoBERTa outperformed RNNs, LSTMs, and BERT.
    RoBERTa was then fine-tuned on Reddit and Wikipedia data for this classification problem. This resulted in a model with a 
    $69\%$ accuracy.\\
    Next, the project attempts to improve this accuracy by including context to aid with the classification process. The context that was included
    was the media attached to the post as well as the comments/threads that the post was a part of. This resulted in around a $20\%$ improvement in
    accuracy.\\
    Finally, the project works on creating a User Interface that allows a user to view the difference in topics they see on social media compared
    to what is commonly available across the whole of the social media site. This allows users to gauge what topics they are interested in. On top
    of this users are able to search for posts relating to any topics of their choice.
\end{abstract}