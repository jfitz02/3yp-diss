\chapter{Conclusions}
\label{ch:conclusions}

The project aimed to:
\begin{itemize}
    \item Classify social media posts
    \item Quantify and compare a users social media feed to the rest of the social media site to identify interests
    \item Build a user interface that allows users to discover what their social media feed says about their interests and how they compare to the rest of the social media site
    \item Build a user interface to allow users to find posts that are dissimilar to the posts they are shown
\end{itemize}

It is clear that all of these aims were met. RoBERTa was fine-tuned for the topic classification task. The fine-tuning was done
using labelled data from Reddit and Wikipedia. Quantification of a set of tweets was done using the mean of the probabilistic outputs
from our model. A user interface was built that is able to show a comparison of topics between a users social media feed and the rest
of the social media site. The user interface also allows users to discover posts that are dissimilar to the posts they are shown.\\
From this, we can conclude that the project was a success.
\section{Future work}
\subsection{Advanced Context Input}
TODO: discuss how we can include context from images and videos better using other Transformer models.
Can also include other aspects of context including: Author, data from linked articles, etc.
\subsection{Chrome Extension}
TODO: take the current Python User interface and turn it into a Chrome extension.
\subsection{Bias Analysis}
TODO: develop a definition of bias, and then explain how we can further our model to analyse bias in social media posts.